\documentclass[conference]{IEEEtran}
\usepackage{cite, hyperref, graphicx, color}
\usepackage[utf8]{inputenc}
\usepackage[brazil, english]{babel}
\title{$f^3$ o $f^{-f, -p, -b}$}
\author{Lucas Tonussi\\lptonussi@gmail.com\\Insegurança em 
Computação\\Universidade Federal de Santa Catarina}

\begin{document}

\maketitle

\begin{abstract}
Estudo aprofundado de \textbf{SHA3}, baseado no material: The National Institute
of Standards and Technology's (NIST) Computer Security Division (CSD) released
its draft FIPS Pub 202 [35 pdf pages] titled "SHA-3 Standard: Permutation -
Based Hash and Extendable-Output Functions".
\end{abstract}

\IEEEoverridecommandlockouts

\begin{keywords}
keccak-f, sha2, sha3, sponge function, hashes, state array, nist, keccak-p, 
step mappings, pré-imagem, colisão.
\end{keywords}

\IEEEpeerreviewmaketitle

\section{Questões}

\begin{enumerate}

\item Definição de função de \underline{hash} criptográfica.

\begin{enumerate}
\item Função de \underline{hash} criptográfica é um tipo de função hash
necessária para aplicações de segurança. É um algoritmo do qual é um problema
intratável de se achar uma informação que mapeia para um resumo
(unilateralidade), ou duas informações que mapeiam para um mesmo resumo (livre
de colisões) (Stallings, pg 314).

\item Para \textbf{funções hash}, a entrada é chamada de mensagem $m$, e a saída
é chamada a mensagem moída ou um valor hash. O tamanho da mensagem pode varias é
claro; mas o tamanho da mensagem moída é fixo. Uma função de hash criptográfico
é uma função hash que é desenhada para prover propriedades especiais, incluindo
resistência a colisão e resistência a bilateralidade (preimage), que são
importantíssimas para muitas aplicações envolvendo segurança da informação.
\end{enumerate}

\item Propriedades.

Requerimentos e Segurança para funções de \underline{hash} criptográfico,
existem alguns porém a questão pede apenas 3  requisitos. (Stallings Página 323
Capítulo 11.3, Tabela 11.1,  Linhas 4, 5, 6).

\begin{enumerate}
\item Preimage resistant (one-way property): $h = H(m),\ h \rightarrow m$ 
\underline{\color{red} \textbf{mas}} \underline{\color{red} \textbf{nunca}} $h 
\leftarrow m$.

\item Second preimage resistant (weak collision resistant): $len(H_1(m_1)) = 
len(H_2(m_2)) = \ldots = len(H_n(m_n))$ \underline{\color{red} \textbf{deve}} 
\underline{\color{red} \textbf{ser}} $len(H(m)) \le len(m)$.

\item Collision resistant (strong collision resistant): $h_1 = H(m_1),\ 
h_2=H(m_2),\ \ldots,\ h_n=H(m_n)\  \vert\ h_1 \not=\ h_2 \not=,\ \ldots,\ 
\not=\ h_n$.
\end{enumerate}

\item Explicar o \textbf{SHA3}.

É um acrônimo para Secure Hash Algorithm-3. É uma família de algoritmos de
\underline{hash} criptográfico. Contém várias restrições nessa família. Os
algoritmos dessa família devem seguir o mesmo arquétipo da família $SHA^3$. Do
contrário não são dessa família. O Arquétipo mais novo prevê suplementar a
famílias mais antigas que já estão depreciadas são elas $SHA^1$ e $SHA^2$. Essa
padronização (arquétipo) é uma nova família de funcionalidades especificadas na
FIPS 180-4\cite{fips}. O SHA-3 é \underline{\textbf{baseada}} no \textsc{KECCAK}
\cite{keccak}, o algoritmo que a NIST (Instituto Nacional de Padrões e
Tecnologia) selecionou como vencedor na competição pública "SHA-3 Cryptographic
Hash Algorithm Competition" \cite{sha3}. Essa família consiste de 4 algoritmos
de função hash criptográfico são eles: SHA3-224, SHA3-256, SHA3-384, e SHA3-512;
dois são funções de vazão estendida são eles: SHAKE128 e SHAKE256 (SHAKE, Secure
Hash Algorithm Keccak).

\item Responder as seguintes questões relacionadas a referência\cite{nist} que
apresenta o $SHA^3$ padronizado pelo NIST.

\begin{enumerate}

\item O que é e para que serve o \textit{State Array} (figura 1)?

\item Como é feita a conversar de strings para \textit{State Arrays}? Mostrar
também um exemplo de vocês, diferente do NIST.

\item Como é feita a conversão de \textit{State Array} para Strings? Incluir um
exemplo de vocês.

\item Explicar os cinco Passos de Mapeamento (NIST, pg 19, 3.2 Step Mappings).
Explicar os algoritmos envolvidos e cada uma das figuras apresentadas (figuras 3
a 6, páginas 19 até 23).

\item Explicar a permutação $KECCAK-p[b,\ n_r]$.

Dado um arranjo de estados, A e também um índice de rodada $i_r$ então a função
de rodada "Rnd" (A subfunção de rodada da permutação $KECCAK -p$) é uma
transformação que resulta da aplicação dos Passos de Mapeamento $\theta,\ \rho,\
\pi,\ \chi,\ \iota$ nessa mesma ordem, resumindo:

$Rnd(A, i_r) = \iota(\chi(\pi(\rho(\theta(A)))), i_r)$

A permutação $KECCAK -p[b, n r ]$ consiste de $n_r$ iteração do procedimento
$Rnd$, como especificado no Algoritmo 7.

Segue o Algoritmo 7: $KECCAK -p[b, n_r](S)$

Entrada: Mensagem $S$ de tamanho $b$; Número de rodadas $n_r$. Saída: Mensagem
$S'$ de tamanho $b$. Algoritmo:

Converter $S$ em Arranjo de Estados $A$, como descrito na Sec. 3.1.2
\cite{nist}.

Para $i_r$ de $2l + 12 - n_r \textbf{ até } 2l + 12 - 1$, $A = Rnd(A,\ i_r)$.

Converter $A$ na string $S'$ de tamanho $b$, como descrito na Sec. 3.1.3.
\cite{nist}

Retornar $S'$.

\begin{enumerate}
\item Compare esta função com a $Keccak -f$.
\end{enumerate}

A subfamília $Keccak -f$ de permutações, originalmente definida em
\cite{bertoni}, é uma \underline{especialização} da subfamília $Keccak -p$ para
o caso $n_r = 12 + 21$.

$$Keccak -f[b] = Keccak -p[b, 12 + 2l]$$

Consecutivamente, a permutação $Keccak -p[1600, 24]$, o qual delineia as 6
funções $SHA^3$, é \underline{equivalente} ao $Keccak -f [1600]$. As rodadas do
$Keccak -f [b]$ são indexadas de $0$ até $11 + 21$. O resultado é indexado
dentre o Passo 2 do Algoritmo 7 pois a rodada do $Keccak -p [b, n_r]$ bate com a
última rodada do $Keccak -f [b]$, e vice versa.

Por exemplo, $Keccak -p[1600, 19]$ é equivalente para com a última rodada de
número 19 do $Keccak -f [1600]$. Similarmente, $Keccak -f [1600]$ é equivalente
para com a última rodada de número 24 do $Keccak -p [1600, 30]$; mas nesse caso,
o rodadas precedentes para o $Keccak -p[1600, 30]$ são indexadas de $-6$ à $-1$.
\item Descrever o \textit{Framework Sponge Construction}.

\begin{enumerate}

\item Explique a figura 7 e o Algoritmo 8.

\end{enumerate}

\item Explique a família de funções esponja Keccak, conforme seção 5.



\item Explique as especificação da função SHA3, conforme Seção 6.

\begin{enumerate}

\item Funções de hash SHA3.

\item Funções de Saída Estendida.

\end{enumerate}

\item Apresente a análise de segurança conforme Apêndice A.1.

\item Gere os seus próprios exemplos (diferente do NIST) conforme Apêndice A.2.

\end{enumerate}

\item Apresente uma implementação do SHA3.

\begin{enumerate}

\item Descreva a implementação.

\item Mostre na implementação onde se da cada passo importante do cálculo do
hash.

\item Execute a implementação passo a passo, mostrando o maior número possível
de saídas.

\end{enumerate}

\item Compare o SHA2, em termos de performance, com os hashes da família SHA2.
Use um mesmo computador e implementações padrão para esta comparação. O
resultado da comparação deve ser em termos de tamanhos de arquivos dos quais
hashes são calculados e quanto tempo para hash (uma média) demora para ser
calculado.

\item Apresente uma crítica ao SHA3.

\begin{enumerate}

\item O que ele é melhor ou diferente em relação a outros hash?

\item Quanto tempo você acha (e por que) o SHA3 será considerado
seguro?

\end{enumerate}

\end{enumerate}

\begin{thebibliography}{9}
\bibitem{nist} NIST Computer Security Division (CSD). SHA-3 Standard:
Permutation-Based Hash and Extendable-Output Functions. 2014. 27p.

\bibitem{fips} Federal Information Processing Standards Publication 180-4,
Secure Hash Standard (SHS), Information Technology Laboratory, National
Institute of Standards and Technology, March 2012,
http://csrc.nist.gov/publications/fips/fips180-4/fips-180-4.pdf.

\bibitem{sha3} The SHA-3 Cryptographic Hash Algorithm Competition, November
2007-October 2012, http://csrc.nist.gov/groups/ST/hash/sha-3/index.html.

\bibitem{keccak} G. Bertoni, J. Daemen, M. Peeters, and G. Van Assche, “Keccak
Specifications,” Submission to NIST (Round 3), January 2011,
http://keccak.noekeon.org/Keccak- submission-3.pdf.

\bibitem{bertoni} G. Bertoni, J. Daemen, M. Peeters, and G. Van Assche, “The K
ECCAK reference, Version 3.0,” January 2011,
http://keccak.noekeon.org/Keccak-reference-3.0.pdf.
\end{thebibliography}

\smallskip

\end{document}
