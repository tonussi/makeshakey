\documentclass[conference]{IEEEtran}
\usepackage{cite, hyperref, graphicx}
\usepackage[utf8]{inputenc}
\usepackage[brazil, english]{babel}
\title{\LARGE SHA}
\author{Lucas Tonussi\\lptonussi@gmail.com\\Insegurança em 
Computação\\Universidade Federal de Santa Catarina}

\begin{document}

\maketitle

\begin{abstract}
Estudo aprofundado de \textbf{SHA3}, baseado no material: The National Institute
of Standards and Technology's (NIST) Computer Security Division (CSD) released
its draft FIPS Pub 202 [35 pdf pages] titled "SHA-3 Standard: Permutation -
Based Hash and Extendable-Output Functions".
\end{abstract}

\IEEEoverridecommandlockouts

\begin{keywords}
keccak-f, sha2, sha3, sponge function, hashes, state array, nist, keccak-p, 
step mappings, pré-imagem, colisão.
\end{keywords}

\IEEEpeerreviewmaketitle

\section{Questões}

\begin{enumerate}

\item Definição de função de \underline{hash} criptográfica.

\begin{enumerate}
\item Função de \underline{hash} criptográfica é um tipo de função hash
necessária para aplicações de segurança. É um algoritmo do qual é um problema
intratável de se achar uma informação que mapeia para um resumo
(unilateralidade), ou duas informações que mapeiam para um mesmo resumo (livre
de colisões) (Stallings, pg 314).

\begin{enumerate}
\item Unidirecionalidade: $h = H(m),\ h \rightarrow m,\
\underline{\textbf{mas}}\ \underline{\textbf{nunca}}\ h \leftarrow m$

\item Compressão: $len(H_1(m_1)) = len(H_2(m_2)) = \ldots = len(H_n(m_n))\
\underline{\textbf{deve}}\ \underline{\textbf{ser}}\ len(H(m)) < 
len(m)$.

\item Primitivas: Calcular $h = H(m)$ usando funções simples.

\item Difusão: o resumo $H(m)$ deve ser uma função complexa de todos os bits da
mensagem $M$.

\item Avalanche: Modifica um só bit da mensagem $M$, então $h = H(m)$ deverá
mudar a metade dos seus bits aproximadamente (50\%).

\item Colisão simples: será computacionalmente intratável, conhecido $M$,
encontrar outro $M’$ tal que $H_1(m_1) = H_2(m_2)$. Isto se conhece como
resistência a colisões.

\item Colisão forte: será computacionalmente difícil encontrar um par $(M_1,\
M_2)$ de forma que $H_1(m_1) = H_2(m_2)$. Isto se conhece como resistência forte
contra colisões.
\end{enumerate}

\end{enumerate}

\item \underline{Propriedades}.

\begin{enumerate}

\item Pré-imagem.

\item Segunda pré-imagem.

\item Colisão.

\end{enumerate}

\item Explicar o \textbf{SHA3}.

\item Responder as seguintes questões relacionadas a referência\cite{nist} que
apresenta o \textbf{SHA-3} padronizado pelo \textbf{NIST}.

\begin{enumerate}

\item O que é e para que serve o \textit{State Array} (figura 1)?

\item Como é feita a conversar de strings para \textit{State Arrays}? Mostrar
também um exemplo de vocês, diferente do NIST.

\item Como é feita a conversão de \textit{State Array} para Strings? Incluir um
exemplo de vocês.

\item Explicar os cinco passos de mapeamento (\textit{Step Mappings}). Explicar
os algoritmos envolvidos e cada uma das figuras apresentadas (figuras 3 a 6).

\item Explicar a permutação \textsc{Keccak-p[b,nr]}.

\begin{enumerate}

\item Compare esta função com a Keccak-f.

\end{enumerate}

\item Descrever o \textit{Framework Sponge Construction}.

\begin{enumerate}

\item Explique a figura 7 e o Algoritmo 8.

\end{enumerate}

\item Explique a família de funções esponja Keccak, conforme seção 5.

\item Explique as especificação da função SHA3, conforme Seção 6.

\begin{enumerate}

\item Funções de hash \textbf{SHA3}.

\item Funções de Saída Estendida.

\end{enumerate}

\item Apresente a análise de segurança conforme Apêndice A.1.

\item Gere os seus próprios exemplos (diferente do NIST) conforme Apêndice A.2.

\end{enumerate}

\item Apresente uma implementação do \underline{SHA3}.

\begin{enumerate}

\item Descreva a implementação.

\item Mostre na implementação onde se da cada passo importante do cálculo do
\underline{hash}.

\item Execute a implementação passo a passo, mostrando o maior número possível
de saídas.

\end{enumerate}

\item Compare o \textbf{SHA2}, em termos de performance, com os
\underline{hashes} da família \textbf{SHA2}. Use um mesmo computador e
implementações padrão para esta comparação. O resultado da comparação deve ser
em termos de tamanhos de arquivos dos quais hashes são calculados e quanto tempo
para \underline{hash} (uma média) demora para ser calculado.

\item Apresente uma crítica ao \textbf{SHA3}.

\begin{enumerate}

\item O que ele é melhor ou diferente em relação a outros \textbf{hashes}?

\item Quanto tempo você acha (e por que) o \textbf{SHA3} será considerado
seguro?

\end{enumerate}

\end{enumerate}

\begin{thebibliography}{9}

\bibitem{nist} NIST Computer Security Division (CSD). SHA-3 Standard:
Permutation-Based Hash and Extendable-Output Functions. 2014. 27p.



\end{thebibliography}

\smallskip

\end{document}
